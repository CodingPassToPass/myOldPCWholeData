(1).
-- image is immutable ( instead you can create a new version of the image that includes the updates).
-- portable
-- alldependencies
-- Docker Image (class:-blueprint or setup or dependencies)(image is immutable)(class)
-- Docker Container (object:-memory or resources occupies or resources use )(instance)
-- IMAGE->Container (Container is instance of image)
-- Docker Hub

(2). CLI commands
-- docker -v
-- docer run --help
-- docker(it gives all information)
-- docker pull hello-world (first container pull)
-- docker run hello-world (first container run) ( if you run without pull, then it firstly check locally, and then go to docker hub and pull it, without pull command)
-- docker push ---image----

mongodb = my-image(take as example)
-- docker run mongodb [ container stop after closing the terminal]
-- docker run -d mongodb (detach)[it will also run after close the terminal]
-- docker run --name gaganshu mongodb(naming of container)[ we can give the name of container]
-- docker run -p 4000:4000 mongodb [we giving port number]  [ first port is for browser and second port is for container]( means(Example:- 2000:4000) agar browser pa port 2000 use karunga then container ka 4000 port access hoga(second port))
    --( first port 4000 is the host port, which is the port of the host machine  that will be exposed outside the environment)
    --( second port 4000 is the container port, which is the port inside container that the application is listening on)
-- docker run -e PORT=4000 -e MONGO_URI=dfsfd -e name=sfdsdf mongodb(we can give multiple env variables)
    --( or, docker run -e PORT=4000 ,MONGO_URI=dfsfd, name=sfdsdf mongodb(my-image) )
    --( or --env as -e)

--docker run --env-file ././fff/env1.txt -e var=3 my-image( we can giveenv file using location)
--docker run --env-file env1.txt, env1.txt,env1.txt  my-image( we can giveenv file using location)

--docker images( get all available images)
--docker rmi imageName ( remove image)
--docker ps (show only running containers)